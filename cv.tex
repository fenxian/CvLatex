\documentclass[[a4paper,12pt]{article}


\usepackage[slantfont,boldfont]{xeCJK}%------------------------------------------ 使用 xeCJK 宏包
\setCJKmainfont{仿宋}%设置正文为宋体
\setCJKmonofont{Adobe Song Std}%设置等距字体
\setCJKsansfont{楷体}%设置无衬线字体
\setCJKfamilyfont{hei}{黑体}

\usepackage[left=2cm,right=2cm,top=1cm,bottom=2cm]{geometry}%------------ 设置页边距

\usepackage{multirow}
\usepackage{tabularx}
\usepackage{setspace}
\usepackage{nopageno}


\usepackage{marvosym,array,etaremune,geometry,ifsym,paralist,fixltx2e,fourier,pifont}
\usepackage{ragged2e,titlesec,xcolor}

\usepackage{hyperref}

\hypersetup
  {
    hidelinks = true             ,
    pdfauthor = Joseph Wright    ,
    pdftitle  = Curriculum Vitae
  }

%\geometry
%  {
%    a4paper         ,
%    nohead          ,
%    nofoot          ,
%    hmargin = 1.5cm ,
%    vmargin = 2cm
%  }
\usepackage{xcolor}
\definecolor{seccolor}{RGB}{155,187,89}
\definecolor{subcolor}{RGB}{146,6,50}
\usepackage{tikz,pgf}
\titleformat{\section}{\Large\bfseries\sffamily}{}{0 em}
  {%
    \begingroup
      \color{seccolor}%
      %\titleline{\leaders\hrule height 0.6 em\hfill\kern 0 pt\relax}%
    \titleline{\tikz\shade[left color=seccolor,right color=white](0,0) rectangle (\textwidth,0.3);\relax}%
    \endgroup
    \nobreak
    \vspace{-1.2 em}%
    \nobreak
  }

\usepackage{titlesec}
\usepackage{titling}
%\titlespacing*{章节命令}{左边距}{上文距}{下文距}[右边距]
\titlespacing*{\section} {0pt}{0.5ex plus 0ex minus .1ex}{0.0ex plus .1ex}
\titleformat{\subsection}{\large\itshape}{}{0 em}{}

%设置题目的字体和字号
\newcommand{\titlefont}{
\setCJKfamilyfont{hei}{黑体}
\fontsize{20}{24}
}



\renewcommand*\arraystretch{1.4}



\newcommand*{\paper}[2]
  {\item \href{http://dx.doi.org/#1}{\ignorespaces#2\unskip.}}
\newcommand*{\papertitle}[1]
  {%
    \begingroup
      \ifluatex
        \addfontfeature{Numbers = Lining}%
      \fi
      \emph{#1}%
    \endgroup
  }

\newlength{\sidewidth}
\newlength{\mainwidth}
\AtBeginDocument
  {%
    \settowidth{\sidewidth}{\textbf{Professional bodies}\hspace{0.75 em}}%
    \setlength{\mainwidth}{\dimexpr\linewidth - \sidewidth\relax}%
  }
\newcommand*{\headline}[1]
  {%
    \hbox{%
      \llap{\hspace*{0.2 em}}%
      \textbf{#1}%
    }%
  }
\newenvironment{CVtable}
  {%
    \begin{tabular}
      {@{}p{\sidewidth}@{\color{subcolor}\aldineright\  }p{\mainwidth}@{}}%
  }
  {\end{tabular}}

\linespread{1.5}

\begin{document}
%\pagestyle{empty}\thispagestyle{empty}

\title{\titlefont \bfseries\sffamily \vspace{-3em} 名字 \ 求职简历 \vspace{-6em}}
%\author{ \vspace{-1cm} 名字 \vspace{-2cm}}
\date{}
\maketitle

\section{基本资料}
\begin{spacing}{1}
\begin{tabularx}{\textwidth}{XXXXXX}
\begin{tabular}{lp{21em}ll}

姓名: & \textbf{\bfseries\sffamily 名 \ 字 \ (男)}  & 学校: & 某某某某大学\\
学历: & 硕 \ 士                                     & 专业: & 计算机科学与技术  \\
电话: & 15111111111                                 & 邮箱: & email@email.com
\end{tabular}
& & &  & & %\raisebox{-5em}{\includegraphics[width=2cm]{picture.jpg}}
\end{tabularx}
\end{spacing}
\vspace{-1em}

\section{教育背景}
\vspace{-1.5em}
\begin{table}[h]
%\centering
\renewcommand\arraystretch{1.5}
\begin{tabularx}{0.9\textwidth}
{  >{\setlength\hsize{1\hsize}\centering}l>{\setlength\hsize{1\hsize}\centering}X
   >{\setlength\hsize{1\hsize}\centering}X>{\setlength\hsize{1\hsize}\centering}X}
  2013- 今    &  某某某某大学     & 计算机科学与技术& 工学硕士 \tabularnewline
  2009-2013   &  某某某某大学      & 信息与计算科学& 理学学士  \tabularnewline
\end{tabularx}
\end{table}
\vspace{-2em}

\section{自我介绍}
\begin{tabularx}{\textwidth}{X}
\hspace{2em}
%\vspace{-2em}
本人性格稳重、待人热情、真诚,对新环境适应能力强,有着乐观向上的生活态度,对IT 行业有着浓厚的兴趣 ...
\end{tabularx}
\vspace{-3em}

%\section{主修课程}
%\linespread{0.5}
%\begin{tabularx}{1\textwidth}{XXX}
%数据结构(84) & 操作系统(92) & 机器学习(90) \\
%高等代数(90) & 数学分析(92) & 数值代数(96)  \\
%\end{tabularx}
%\linespread{1}

\section{专业技能}

%\raggedright
%\begin{itemize}
%\setlength{\itemsep}{1pt}
%   \item \textbf{\bfseries\sffamily 资格证书} \\
%   证券从业资格 \\
%   全国计算机等级考试二级证书 \\
%   正在考取会计从业资格考试
%   \item \textbf{\bfseries\sffamily 计算机使用} \\
%   熟练使用Office (Word, Excel, PPT)办公软件 \par
%   学习过Matlat/SPSS/SAS/R等专业软件,这些基础有助于进一步学习提高,适应公司业务需求
%\end{itemize}

\begin{itemize}
\setlength{\itemsep}{0pt}
%\setlength{\parsep}{100pt}
%\setlength{\leftmargin}{0pt}
%\setlength{\parskip}{0pt}
\item 熟练掌握C++,熟悉SLT模板程序库使用,了解Boost库
\item 掌握OpenCV开源视觉图像库,QT框架的使用
\item 熟练掌握Matlab;掌握Python的使用;使用过Java,Shell等语言
\item 具有良好的数据结构和算法基础
\item 了解常用的机器学习算法和图像处理算法
%\item 了解网络编程、多线程编程等
%\item 熟悉Socket编程%,多线程编程
\item 了解数据库、SQL语言等相关技术
%\item 开发环境/工具:Emacs(熟练),VS(熟练),Git(熟练),CMake(熟练),MySql(一般)
\end{itemize}
\vspace{-3em}

\section{在校实践经验}
\begin{enumerate}
\item \textbf{\bfseries\sffamily 某算法的研究与实现} \\
该项目是 ...

\item \textbf{\bfseries\sffamily 图像浏览/处理软件} \\
该项目包含两项内容 ... 

\item \textbf{\bfseries\sffamily 某算法研究与实现} \\
该项目正在进行中 ...

\end{enumerate}
\vspace{-2em}

\section{奖励及证书}
\begin{itemize}
\setlength{\itemsep}{0pt}
\item 北某某奖学金 \par
\item 北某某奖学金 \par
\item 山某某奖学金 \par
\end{itemize}
\vspace{-2em}

\section{兴趣爱好}

\begin{itemize}
\setlength{\itemsep}{0pt}
\item 喜欢关注社会热点新闻、关注技术热点、行业动态 \par
\item 喜欢读书,尤其是纪实类、人物传记类书籍 \par
\item 爱好运动,喜欢徒步旅行、登山等 \par
\end{itemize}


\end{document}
